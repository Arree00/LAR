\documentclass[a4paper,titlepage,12pt]{article} %titlepage,twocolumn ,draft
\usepackage[latin1]{inputenc}
\usepackage{a4wide}
\usepackage{amsmath}
\usepackage{amssymb}
%\usepackage[utf8]{inputenc}
%\usepackage[swedish]{babel}
\usepackage[english, swedish]{babel}
\usepackage{graphicx}
\usepackage{subfigure}
\usepackage{float}
\usepackage[english]{varioref}
\usepackage{moreverb}


%SIDHUVUD OSV 
%\usepackage{fancyhdr}

%\pagestyle{fancy}
%\fancyhf{}
%\fancyhead[LE,RO]{\textsc{Sidhuvud till h�ger}}
%\fancyhead[RE,LO]{\textsc{Sidhuvud till v�nster}}
%\fancyfoot[CE,CO]{\leftmark}
%\fancyfoot[LE,RO]{\thepage}


%*******************************************************************************
\title{ \includegraphics[height=2.5cm]{figures/baldergymnasiet_logo} \\
        \mbox{} \\
        \Huge{Projektrapport: Ert projektnamn} \\
        \normalsize
        \mbox{} \\
        \textsc{Teknik.Spec - Mekatronik - HT18} }
\author{Elev Elevsson \\elev007@skelleftea.se 
		\and Elev Johansson \\elev008@skelleftea.se 
		\and Johan Elevsson \\ elev009@skelleftea.se}
\date{\today}
%*******************************************************************************
\begin{document}
\selectlanguage{swedish}
\maketitle
\clearpage
\newpage

% %%%%%%%%%%%%%%%%%%%%%%%%%%%%%%%%
% H�r kan en sammanfattning finnas. Den skrivs allts� f�re % inneh�llsf�rteckningen
% %%%%%%%%%%%%%%%%%%%%%%%%%%%%%%%%
\pagenumbering{gobble}
\tableofcontents
\newpage
\pagenumbering{arabic}

%*******************************************************************************
\section{Inledning}
%*******************************************************************************
H�r beskriver ni er konstruktion och introducerar l�saren till de problem ni st�llts inf�r vid konstruktionen.


%*******************************************************************************
\section{Mekanisk konstruktion }
%*******************************************************************************
H�r beskriver ni er mekaniska konstruktion. Anv�nd visualisering (principskiss, bilder p� delar av systemet, m.m) om ni tycker att det beh�vs f�r att beskriva mekaniken. 

\subsection{St�lldon}
Ex: En stegmotor har anv�nts f�r att....


%*******************************************************************************
\section{Elektronik}
%*******************************************************************************
Beskrivning av den elektronik som anv�nds i systemet. Anv�nd beskrivande bilder om n�dv�ndigt.

\subsection{Styrkort}
Ex: Ett XXXX anv�nds som central styrenhet....

\subsection{Sensorer}
Ex: Tv� sensorer har anv�nts i system en.....den m�ter XXX genom att....I konstruktionen har sensorn monterats h�r och h�r....eftersom...


%*******************************************************************************
\section{Styrning}
%*******************************************************************************
Beskriv hur styrningen fungerar. Anv�nd g�rna blockschema, eller andra visualiseringstekniker f�r att tydligg�ra hur styrningen fungerar.\\
\\
Ex: XXXX-sensorn l�ser kontinuerligt avst�ndet till......avst�ndsinformationen anv�nds sedan f�r att styra armen....osv osv


%*******************************************************************************
\section{Resultat}
%*******************************************************************************
Ber�tta hur bra systemet fungerade.

%*******************************************************************************
\section{Diskussion}
%*******************************************************************************
Diskutera med utg�ngspunkt i ert tidigare projektf�rslag resultatet, konstruktionen, de problem ni st�tt p� och hur ni l�st dessa. St�mde dessa �verrens med det ni t�nkte redan innan?

\subsection{L�rdomar}
Vad anser ni att ni har l�rt er under projektets g�ng?

\newpage

%*******************************************************************************
\end{document}
%*******************************************************************************
